\documentclass[poll_tutorial_format]{subfiles}
\begin{document}
	\maketitle
	\section{PT Week 4 Expectation (cases with (Discrete) random variables)}
	
	\subsection{Set things up}
	\label{sec:set-things-up}
	
	
	
	\setcounter{theorem}{-1}
	
	\begin{exercise}
		Have you helped your neighbors to set up their polleverywhere app? 
		\begin{enumerate}
			\item Yes
			\item No
		\end{enumerate}
	\end{exercise}
	
	\subsection{Real questions}
	\label{sec:start-real-questions pt week 4}
	\begin{exercise}
		Suppose we have a sample space $S$ with a subset/event $A$. Denote $X$ the random variable such that $X(s)=1$ for $s\in A$ otherwise $0$ 
		Which one of these statements could be incorrect:%TODO $P(X=0|A)=1$ should be 0. 
		\begin{enumerate}
			\item $EX=P(X=1)$.
			\item $EX=P(A)$.
			\item The support of $X$ is $\{0,1\}$.
			\item $P(X=0|A)=1$. 
		\end{enumerate}
	\end{exercise}
	
	
	\begin{exercise}
		(Coin tossing problem) A fair coin is flipped two times, consider the sample space $S=\{HH, HT, TH, TT\}$ (H stands for head, and T stands for tail). Denote r.v. $X$ the number of heads within two tosses.
		Choose one of these answers that is incorrect: %TODO ans:$EX^2=(\sum_{i=0}^2 iP(X=i))^2$ (slightly tricky as $EX^2=\sum_{i=0}^9 iP(X^2=i)$ include some terms that are simply zero).
		\begin{enumerate}
			\item $EX=\sum_{i=0}^2 iP(X=i)$.
			\item $EX^2=\left(\sum_{i=0}^2 iP(X=i) \right)^2$ 
			\item $EX$ is a real number.
			\item $EX^2=\sum_{i=0}^2 i^2P(X=i)$
			\item $EX^2=\sum_{i=0,1,4,9} iP(X^2=i)$
		\end{enumerate}
	\end{exercise}
	
	
		
	\begin{exercise}
		(Coin tossing problem) A fair coin is flipped two times, consider the sample space $S=\{HH, HT, TH, TT\}$ (H stands for head, and T stands for tail). Denote r.v. $X$ the number of heads within the two tosses, and $Y$ the number of tails within the two tosses.
		Choose one of these answers that is incorrect: %TODO ans:$EX^2=(\sum_{i=0}^2 iP(X=i))^2$ (slightly tricky as $EX^2=\sum_{i=0}^9 iP(X^2=i)$ include some terms that are simply zero).
		\begin{enumerate}
			\item $EaX+b=aEX+b$ for arbitrary numbers $a$ and $b$ 
			\item $E(X^2 +Y^2)=EX^2 +EY^2$
			\item $E(aX+bY)=aEX +bEY$
			\item $EX=2-EY$ 		
			\item $X$ and $Y$ is independent.
		\end{enumerate}
	\end{exercise}
	
	
	\begin{exercise}
		Choose one of these answers that is incorrect: %TODO ans:Expectations of any discrete random variables always exist. (Consider the counter example where X takes values i with probability ($\frac{\pi^2}{6}$)^{-1}$\frac{1}{i^2}$) 
		\begin{enumerate}
			\item The expected value of $X$ is a weighted average of the possible values that $X$ can take on, weighted by their probabilities
			\item The expectation of $X$, if exists, is nothing but a number.
			\item  The expectation of a discrete random variable, $X$, is defined as $EX=\sum_{x\in \textit{Support}(X)} xP(X=x)$.
			\item Expectations of any discrete random variables always exist.  
		\end{enumerate}
	\end{exercise}
	
	
	
	\begin{exercise}
		Choose one of these answers that is incorrect (or the last choice if none is incorrect):%TODO ans:If $X$ and $Y$ have different distributions, then they have different expected values. whose incorrectness is also why it may be a bad idea to replace the r.v. with its expected value.
		\begin{enumerate}
			\item $E(X)$ depends only on the distribution of $X$: If $X$ and $Y$ are discrete r.v.s with the same distribution, then $E(X) = E(Y)$ (if either side exists).
			\item The support of $X$ is $a_1, a_2, a_3, \dots$, and then $EX=\sum_{i=1}^\infty a_i P(X=a_i)$.
			\item If $X$ and $Y$ have different distributions, then they have different expected values. 
			\item One only needs to know CDF or PMF of a discrete random variable $X$ to calculate the expected value of $X$.
			\item All of the above are correct. 
		\end{enumerate}
	\end{exercise}
	
	
	\begin{exercise}
		Choose one of these answers that is incorrect:%TODO ans:$Ee^X=e^{EX}$ 
		\begin{enumerate}
			\item $Ee^X=e^{EX}$ for an arbitrary r.v. $X$.  
			\item $E(X+Y)=EX+EY$
			\item $E(X+Y)^2 =EX^2 +EY^2 +2EXY$  
			\item $V(X)=E\left(X-EX \right)^2$.  
		\end{enumerate}
	\end{exercise}
	
	
		
	\begin{exercise}
		Choose one of these answers that is incorrect:%TODO ans:$Ee^X=e^{EX}$ 
		\begin{enumerate}
			\item $V(X)=E\left(X-EX \right)^2$  
			\item $V(X)=EX^2 -\left(EX \right)^2$
			\item $EX^2 \geq \left(EX \right)^2$  
			\item The variance of a r.v. $X$, denoted by $V(X)$, could be a negative value.
		\end{enumerate}
	\end{exercise}
	
	
%	indicator question
	\begin{exercise}
		Denote $X$ a discrete random variable following the Discrete Uniform distribution with support $\{-2,-1,1,2\}$.
		Choose one of these answers that is incorrect: %TODO ans: The support of $f(X)$ with $f(x)=x^2$ is $\{-1,-2,1,2\}$  
		\begin{enumerate}
			\item The PMF function $P(X=a)=1/4$ for $a\in \{-2,-1,1,2\}$ otherwise 0 describes the how the probability is distributed among events generated by $X$. 
			\item The support of $X^2$ is $\{1,4\}$
			\item The support of $f(X)$ with $f(x)=x^2$ is $\{-1,-2,1,2\}$  
			\item The CDF function of $X$ describes how the probability is distributed among events generated by $X$.  
			\item The first and the forth choices.
		\end{enumerate}
	\end{exercise}
	

%	LOTUS	
	\begin{exercise}
		Let $X_1,\cdots , X_n$ follow independent Bernoulli distribution with the same successful rate $1/2$, 
		Choose one of these answers that is incorrect: %TODO ans:The support of $\sum_{i=1}^n X_n$ is $\{1,2,3,4,...,n\}$.
		\begin{enumerate}
			\item The PMF of $X_i$ is $P(X_i=1)=1/2$ and $P(X_i=0)=1/2$ otherwise 0.
			\item The CDF of $X_i$ is $F(x)=0$ for $x<0$; $F(x)=1/2$ for $0\leq x <1$; and $F(x)=1$ for $x\geq 1$.  
			\item $\sum_{i=1}^n X_n$ follows  Bin(n,1/2).
			\item The support of $\sum_{i=1}^n X_n$ is $\{1,2,3,4,...,n\}$.
		\end{enumerate}
	\end{exercise}
	
	
% Variance calculation question	
	\begin{exercise}
		(Coin tossing problem) A fair coin is flipped two times, event A represents two tosses landed head and B represents the event that the first toss landed tail. 
		Choose one of these answers that is incorrect: %TODO ans: $P(A|B^c)=P(B^c|A)$
		\begin{enumerate}
			\item $P(B\cup B^c )=P(B)+P(B^c)$
			\item $P(B\cup B^c |A)=P(B|A)+P(B^c|A)$
			\item $P(A|B^c)=P(B^c|A)$
			\item $P(A|B)=0$
			\item $P(A|B^c)=P(B)$
		\end{enumerate}
	\end{exercise}
	
	 
	
	
	
\end{document}
