\begin{exercise}\textbf{[BH.7]}
	Two chess players, A and B, are going to play 7 games. 
		\begin{enumerate}
		\item First answer the original questions in your own words.
		\item Extensions based on this story where we visit the concept of conditional probability and independence.
		\begin{enumerate}
					\item How many possible outcomes, denoted by N, in total? (If you could not derive the number, just use the term N to represent the number if it is needed in later exercise.)
			\item Can you propose one sample space if we care for 
			\begin{enumerate}
				\item only the outcome of the first game;
				\item the outcome of all outcomes
			\end{enumerate}
			and how would you express the outcome of the first game using the sample space you propose for the latter case.
			\item Now, we assume that $A$ and $B$ are such players that all three possible outcomes in one game are equally likely, and outcomes in the seven games are independent from each other (i.e., the outcomes of the seven games are like throwing a 3-sided fair tie for seven times). 
			\begin{enumerate}
				\item What is the probability that overall player A ends up with 3 wins, 2 draws, and 2 losses?
				\item What is the probability that that overall player A ends up with 7 draws? What is the connection between this value and the probability value that A ends up with a draw in one game, e.g., in the first game.
				\item What is the probability that A ends up with 3 wins, 2 draws, and 2 losses given that A wins the first three rounds? Provide two different approaches. (Hint: counting, and conditional probability!) 
			\end{enumerate}
		\end{enumerate}
		\end{enumerate}
\end{exercise}
