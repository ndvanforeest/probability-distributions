\begin{exercise}
	\textbf{[BH . 37]}  $T\sim $   Expo$(\lambda)$ is the time until a radioactive particle decays.
	\begin{enumerate}
		\item First answer the original questions in your own words.		
		\item Additionally, consider now that $T_1, \ldots, T_n \sim$ $\operatorname{Expo}(\lambda/n)$ to make sure the total expected decays is fixed, what is the probability that there happen to be $k$ decays in the time $[0,t]$, what happens to this probability value if we let $n\rightarrow\infty$ (where you can use the fact that $\frac{n!}{n^k (n-k)!} \rightarrow 1$ and $n^k (1-e^{-a/n})^k \rightarrow a^k$)? 
	\end{enumerate} 

\end{exercise}